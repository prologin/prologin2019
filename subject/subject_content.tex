% SPDX-License-Identifier: GPL-2.0-or-later
% Copyright 2019 Cyril Amar

\newpage

\section{Le début des ennuis}

Félicitations ! Votre dernière promotion au titre de Responsable est bien méritée. Vous voici donc chargé de superviser l'extraction de minerais rares d'une mine standard. Malheureusement, la Direction ne vous a pas accordé tous les moyens nécessaires\ldots{}

Passons donc en revue vos nouvelles conditions d'exercice.

\subsection{La mine (standard)}
Vous intervenez sur une mine (standard) déjà partiellement exploitée. Les rapports géologiques font état de nombreuses cavités naturelles sur le site.

\subsection{Le minerai (standard)}
\textit{Le granit est une roche magmatique plutonique à structure grenu, c'est-à-dire entièrement cristalisée, formée par le refroidissement lent et en profondeur d'un magma issu de la fusion partielle de la croûte continentale. C'est une roche acide composée essentiellement de quartz et de mica. Bla bla bla\ldots{}}

La région est riche en granit (standard), mais il existe des veines de minerai de plus forte valeur. Vous consulterez avec attention le rapport de votre géologue en chef qui reprend les différents minerais présents sur site.

\subsection{Les Nains (standards)}
\textit{Les Nains sont des être humanoïdes de petite taille avec une barbe, deux bras, deux jambes et un casque.}

La Direction vous a affecté la meilleure catégorie de personnel pour l'exploitation, vous êtes donc à la tête d'une équipe de Nains(standards). Le Manuel relatif à l'Exploitation Minière, annexé en fin de document, vous donnera de plus amples détails. Sachez toutefois que les Nains ont certaines\ldots{} particularités, comme des Points d'Action, de Déplacement et de Vie.

\subsection{C'est qui, lui ?}
Lui ? Ah ! Lui !

Hum, c'est vrai qu'on aurait pu commencer par là. Pour doubler l'activité et donc le profit, il a été décidé de doubler les effectifs\footnote{Puisqu'il est bien connu que neuf femmes font un bébé en un mois\ldots{}}. Il y a donc un autre Responsable, avec sa propre équipe.

Vous êtes heu\ldots{} en concurrence directe, car la Direction évaluera vos rendements respectifs.

\newpage
\section{Rapport géologique}
\subsection{Le granit (standard)}
\textit{Le granit est une roche magmatique plutonique à structure grenu, c'est-à-dire entièrement cristalisée, formée par le refroi\ldots{}}

Ah on vous l'a déjà dit ? Bon bon bon\ldots{} j'accélère alors.

Ce qu'on ne vous a probablement pas dit, c'est que le granit (standard) a une valeur faciale de très exactement 0 pièce d'or, et requiert un tour pour être cassé.


\subsection{Le minerai}
Du coup, en plus du granit (standard), on a identifié //FIXME types de minerai. 
Leur valeur en pièces d'or varie, de même que le nombre de coups de pioche nécessaire pour les extraire.
J'ai pas les valeurs exactes avec moi, mais un de mes sous-fi\ldots{} collègues vous les donnera quand vous serez sur place.

Par contre, j'peux vous montrer à quoi ça ressemble :
\begin{center}
        \begin{tabular}{|l|c|}
                \hline
                Minerai & Échantillon \\
                \hline
                Fer & //FIXME \\
                Cuivre & //FIXME \\
                \hline
                Granit & //FIXME \\
                \hline
        \end{tabular}
\end{center}

\subsection{Divers}
Oui, j'ai encore des bricoles à vous dire, mais pas assez pour en faire des chapitres complets. Vous avez déjà écrit des rapports, vous voyez ce que je veux dire.

Bref. Faites gaffe aux trous en creusant. Y'a pas mal de cavités naturelles sur ce terrain. C'est pratique pour descendre plus rapidement, mais n'oubliez pas que plus vous chutez de haut plus ça fait mal en arrivant en bas. Normalement dans le Manuel d'Exploit' ils vont expliqueront tout ça.

Ah, et une dernière chose. On a trouvé de l'obsidienne. Vous prenez pas la tête avec, les Nains n'arriveront pas à la miner.

\newpage
\section{Manuel relatif à l'Exploitation Minière}
\textit{Version 13, édition 37, opus 42}

\subsection{Reconnaissance du terrain}
Le standard\footnote{héhéhé} de la Compagnie est d'exploiter les gisements depuis la surface, en creusant verticalement. L'exploitation à flanc de montagne n'est pas pratiquée, sauf exception validée par la Direction.

Le terrain est donc globalement plat, et l'exploitation se fait globalement verticalement.

Conformément aux article 16 et suivants de la Convention Collective applicable, une taverne est présente en surface de chaque mine. Les mesures de sécurité prévoient que les mineurs sans affectation doivent y être présents en permanence. Par dérogation au Code du Travail, la consommation de bière pendant les horaires de travail est autorisée, en vertu de ses propriétés médicales exceptionnelles : chaque mineur se présentant à la taverne récupère instantanément la totalité de ses Points de Vie.

La taverne abrite également une délégation du Département Trading, qui collecte le minerai extrait.

\subsection{Cordages}
La dotation (standard) inclut des cordes (standard) illimitées. Il convient de disposer une poulie en tête de cordage pour faciliter les actions sur la corde.

Une fois la poulie posée, la corde descend jusqu'au prochain bloc.

Poser une corde est une action complexe, qui nécessite la totalité des Points d'Action.

Les cordes sont actionnées par un personnel minier dédié, qui aura les deux pieds au sol. Il n'est pas nécessaire de se trouver à la poulie pour actionner une corde. Cela consomme des Points d'Action.


\subsection{Déplacements}
Le personnel minier peut se déplacer dans deux dimensions : verticalement et transversalement.

Un déplacement n'est possible que vers une position libre\footnote{C'est évident mais ça va toujours mieux en le disant.}, et nécessite des Points de Déplacement.

\subsection{Cas particulier des déplacements verticaux}
Dans le cas particulier des déplacements verticaux, il est rappelé que la gravité existe, et qu'elle attire inexorablement les corps vers le bas.

Les déplacements (dans les deux sens) peuvent être réalisés :
\begin{itemize}
        \item en grimpant à la paroi
        \item en grimpant à une corde
        \item en s'accrochant à une corde qui sera ensuite manipulée par un personnel en tête de corde.
\end{itemize}

\subsection{Traitement du minerai}
Les minerai précieux doivent être extraits par du personnel qualifié. L'extraction consomme des Points d'Action, et peut nécessiter plusieurs coups de pioche en fonction du minerai.

Il est possible de paralléliser l'extraction avec plusieurs mineurs, de manière à accélérer celle-ci.

Une fois extrait, le minerai est stocké temporairement par le personnel. Afin d'être pris en compte par le Département Trading, le minerai doit être rapporté à la taverne.

\subsection{Cas particulier des rencontres malveillantes}
Il est rappelé à l'ensemble du personnel que la pioche (standard) prévue à la dotation (standard) est également une arme de quatrième catégorie. Son usage à cette fin est explicitement autorisé, à la discrétion de chacun. Dans ce cas, la pioche se manie de manière identique à son utilisation standard.

Un soin particulier sera accordé aux points de vie des intervenants. Lorsque ce compteur arrive à zéro, un Point de Destin est automatiquement utilisé. Le mineur ainsi ressucité regagne la mine dans la taverne du chantier au prochain tour.

