% SPDX-License-Identifier: GPL-2.0-or-later
% Copyright 2019 Cyril Amar

\newpage

\section{Le début des ennuis}

  Félicitations! Votre dernière promotion au titre de nain responsable de la
  mine E corp est bien méritée. Vous voici donc chargé de superviser
  l'extraction de minerais rares d'une mine (standard). Malheureusement, la
  direction ne vous a pas accordé tous les moyens nécessaires\ldots{}

  Passons donc en revue vos nouvelles conditions d'exercice.

\subsection{La mine (standard)}

  Vous intervenez sur une mine déjà partiellement exploitée. Les rapports
  géologiques font état de nombreuses cavités naturelles sur le site.

\subsection{Le minerai (standard)}

  \textit{Le granit est une roche magmatique plutonique à structure grenu,
  c'est-à-dire entièrement cristallisée, formée par le refroidissement lent et
  en profondeur d'un magma issu de la fusion partielle de la croûte
  continentale.  C'est une roche acide composée essentiellement de quartz et de
  mica.}

  La région est riche en granit (standard), mais il existe des veines de
  minerai de plus forte valeur. Vous consulterez avec attention le rapport de
  votre géologue en chef qui reprend les différents minerais présents sur site.

\subsection{Les nains (standards)}

  \textit{Les nains sont des être humanoïdes de petite taille avec une barbe,
  deux bras, deux jambes, un casque et généralement une ou plusieurs choppes de
  bières dans les mains.}

  La direction vous a affecté la meilleure catégorie de personnel pour
  l'exploitation, vous êtes donc à la tête d'une équipe de nains. Le manuel
  relatif à l'exploitation minière, annexé en fin de document, vous donnera de
  plus amples détails. Sachez toutefois que les nains ont certaines\ldots{}
  particularités, comme des \textit{points d'action}, \textit{de déplacement}
  et \textit{de vie}.

  Pour des questions de restrictions budgétaires, le nombre de nains
  (standards) est limité.

\subsection{C'est qui, lui?}

  Lui? Ah! Lui!

  Hum, c'est vrai qu'on aurait pu commencer par là\ldots{} Pour doubler
  l'activité et donc le profit, E corp a  décidé de doubler les
  effectifs\footnote{Puisqu'il est bien connu que neuf femmes font un bébé en
  un mois\ldots{}}. Il y a donc un autre Responsable, avec sa propre équipe.

  Vous êtes heu\ldots{} en concurrence directe, car la direction évaluera vos
  rendements respectifs. Certaines équipes ont rapporté quelques incidents où
  certains nains auraient été blessé dans les mines, plus précisément certains
  auraient présenté des signe de contusion de pioches. Soyez
  fairplays\footnote{Et rentables.}.

\newpage

\section{Rapport géologique}

\subsection{Le granit (standard)}

  \textit{Le granit est une roche magmatique plutonique à structure grenu,
  c'est-à-dire entièrement cristallisée, formée par le refroi\ldots{}}

  Ah on vous l'a déjà dit? Bon bon bon\ldots{} j'accélère alors.

  Ce qu'on ne vous a probablement pas dit, c'est que le granit (standard) a une
  valeur de très exactement 0 pièce d'or\footnote{Peut mieux faire en terme de
  rentabilité.}, et requiert un coup de pioche pour être cassé.

\subsection{Le minerai}

  Du coup, en plus du granit (standard), on a identifié 6 types de minerai.  Le
  marché pouvant être particulièrement volatile, leurs valeur en pièces d'or
  varie, de même que le nombre de coups de pioche nécessaire pour les extraire.
  J'ai pas les valeurs exactes avec moi, elle dépendent de la mine en question,
  mais un de mes sous-fi\ldots{} collègues vous les donnera quand vous serez
  sur place.

  Par contre, je peux vous montrer à quoi ça ressemble:
  \begin{center}
    \begin{tabular}{|l|Sc|}
      \hline
      Minerai & Échantillon \\
      \hline
      Obsidienne & \cincludegraphics[width=0.5cm]{frames/obsidian.png} \\
      \hline
      Granit & \cincludegraphics[width=0.5cm]{frames/granit.png} \\
      \hline
      Charbon & \cincludegraphics[width=0.5cm]{frames/coal.png} \\
      \hline
      Fer & \cincludegraphics[width=0.5cm]{frames/iron.png} \\
      \hline
      Or & \cincludegraphics[width=0.5cm]{frames/or.png} \\
      \hline
      Diamants & \cincludegraphics[width=0.5cm]{frames/diamonds.png} \\
      \hline
      Émeraudes & \cincludegraphics[width=0.5cm]{frames/emerauld.png} \\
      \hline
      Rubis & \cincludegraphics[width=0.5cm]{frames/ruby.png} \\
      \hline
    \end{tabular}
  \end{center}

\subsection{Divers}

  Oui, j'ai encore des bricoles à vous dire, mais pas assez pour en faire des
  chapitres complets. Vous avez déjà écrit des rapports, vous voyez ce que je
  veux dire.

  Bref. Faites gaffe aux trous en creusant. Y'a pas mal de cavités naturelles
  sur ce terrain. C'est pratique pour descendre plus rapidement, mais n'oubliez
  pas que plus vous chutez de haut plus ça fait mal en arrivant en bas.
  Normalement dans le manuel d'exploit' ils vont expliqueront tout ça.

  Ah, et une dernière chose. On a trouvé de l'obsidienne. Vous prenez pas la
  tête avec, les nains n'arriveront pas à la miner.

\newpage

\section{Manuel relatif à l'exploitation minière}

  \textit{Version 13, édition 37, opus 42}

\subsection{Reconnaissance du terrain}

  Le standard\footnote{héhéhé} de la compagnie est d'exploiter les gisements
  depuis la surface, en creusant verticalement. L'exploitation à flanc de
  montagne n'est pas pratiquée, sauf exception validée par la direction.

  Le terrain est donc globalement plat, et l'exploitation se fait globalement
  verticalement.

  Conformément aux article 16 et suivants de la Convention Collective
  Applicable, une taverne est présente en surface de chaque mine. Les mesures
  de sécurité prévoient que les mineurs sans affectation doivent y être
  présents en permanence. Par dérogation au code du travail, la consommation de
  bière pendant les horaires de travail est autorisée, en vertu de ses
  propriétés médicales exceptionnelles : chaque mineur se présentant à la
  taverne récupère instantanément la totalité de ses points de vie\footnote{Ce
  type de réaction n'as été observé dans nos laboratoires que chez les nains
  standards adultes.}.

  La taverne abrite également une délégation du département trading, qui
  collecte le minerai extrait.

\subsection{Cordages}

  Le matériel des nains inclut des cordes (standards) illimitées. Pour les
  utiliser il convient de disposer une poulie en tête de cordage, cela
  permettra les actions sur la corde.

  Ces cordes permettent au nains de se déplacer verticalement plus rapidement
  qu'en s'agrippant aux parois et plus surement qu'en chutant.

  Une fois la poulie posée, la corde descend jusqu'au prochain bloc. En
  revanche, poser une corde est une entreprise complexe, qui nécessite la
  totalité des \textit{points d'action} de toute l'équipe combinée.

  De plus les cordes peuvent être actionnées par un autre nain qui n'est pas
  sur la corde, s'il a les deux pieds au sol. En consommant des points
  d'actions le nain peut actionner la corde dans un sens ou dans un autre ce
  qui ralentira ou accélérera le mouvement du nain sur la corde.

\subsection{Déplacements}

  Le personnel minier peut se déplacer dans deux dimensions: verticalement et
  transversalement, en marchant au sol, en s'agrippant à la paroi, en
  descendant le long d'une corde ou en chutant.

  Un déplacement n'est possible que vers une position libre\footnote{C'est
  évident mais ça va toujours mieux en le disant.}, et nécessite des
  \textit{points de déplacement}.

  Il est également à noter que la promiscuité ne pose pas de problème aux
  mineurs\footnote{C'est un critère de recrutement.}, ils peuvent donc se
  trouver sur une même position. Il est néanmoins nécessaire de se coordonner
  un minimum avant: ainsi seuls les mineurs \textbf{d'une même équipe} se
  tolèrent entre eux.

  Enfin, de par leur activité physique intense, les mineurs ont les épaules
  larges. Ils peuvent sans problème supporter le poids d'un autre mineur. En
  application du paragraphe précédent, un mineur ne gardera jamais un mineur
  qui est dans son équipe sur ses épaules car ils peuvent se coordonner afin de
  se tenir sur la même case.

\subsection{Cas particulier des déplacements verticaux}

  Dans le cas particulier des déplacements verticaux, il est rappelé que la
  gravité existe, et qu'elle attire inexorablement les corps vers le bas. Les
  collisions avec le sol entraînent des dégâts exponentiels avec la hauteur de
  chute, selon la formule ci-dessous.

  \[
    \text{Dégâts} = 
    \left\{
      \begin{array}{l l}
        0         & \quad \text{pour $h < 4$}  \\
        2^{h-4} & \quad \text{pour $h >= 4$} \\
      \end{array}
    \right.
  \]

  avec $h$ la hauteur de chute.

  Les déplacements (dans les deux sens) peuvent être réalisés :

  \begin{itemize}
    \item en grimpant à la paroi (relativement lent)
    \item en grimpant à une corde (normal)
    \item en s'accrochant à une corde qui sera ensuite manipulée par un
      personnel en tête de corde. (rapide mais dangereux et à sens unique)
  \end{itemize}

\subsection{Traitement du minerai}

  Les minerai précieux doivent être extraits par du personnel qualifié.
  L'extraction consomme des points d'action, et peut nécessiter plusieurs coups
  de pioche en fonction du minerai extrait.

  Comme les nains n'ont pas de problèmes à se tenir côte à côte, il est
  possible de paralléliser l'extraction avec plusieurs mineurs, de manière à
  accélérer celle-ci.

  Une fois extrait, le minerai est stocké temporairement par le personnel. Afin
  d'être pris en compte par le département trading (et être mis en
  sécurité\ldots{}), le minerai doit être rapporté à la taverne.

\subsection{Cas particulier des rencontres malveillantes}

  Il est rappelé à l'ensemble du personnel que la pioche (standard) qui est
  donnée à chaque nains, peut également être considéré comme une arme de
  quatrième catégorie\ldots{} Son usage à cette fin est explicitement ignoré par la
  direction, à la discrétion de chacun. Dans ce cas, la pioche se manie de
  manière identique à son utilisation standard. À noter que si plusieurs nains
  sont sur la même position et que l'un d'entre eux est attaqué, ils sont tous
  blessé\footnote{Striiike!} !

  Un soin particulier sera accordé aux \textit{points de vie} des intervenants.
  Lorsque ce compteur arrive à zéro, un point de destin\footnote{La direction
  assure qu'un nain aura toujours un point de destin disponible pour lui dans
  cette situation, au risque d'un prélèvement de salaire.} est automatiquement
  utilisé. Le mineur ainsi ressuscité regagne la mine dans la taverne du
  chantier à son prochain tour, mais ayant perdu tout son butin sur
  place\ldots{}
