% SPDX-License-Identifier: GPL-2.0-or-later
% Copyright 2019 Cyril Amar

\newpage

\section{Le début des ennuis}

  Félicitations! Votre dernière promotion au titre de nain responsable de la
  mine au P.R.O.L.O.G.I.N. (Profits Rapides Organisés Logiquement, Obtenus Grâce à des Individus Nanesques.) est bien méritée. Vous voici donc chargé de superviser
  l'extraction de minerais rares d'une mine (standard). Malheureusement, la
  direction ne vous a pas accordé tous les moyens nécessaires\footnote{
  Restriction budgétaire, crise économique, augmentation des taxes\ldots{}}
  \ldots{}

  Passons donc en revue vos nouvelles conditions d'exercice.

\subsection{La mine (standard)}

C'est une mine. Parfois il y a des tunnels déjà creusées,  d'autres fois non.
  
\subsection{Le minerai (standard)}

  \textit{Le granit est une roche magmatique plutonique à structure grenue,
  c'est-à-dire entièrement cristallisée, formée par le refroidissement lent et
  en profondeur d'un magma issu de la fusion partielle de la croûte
  continentale.  C'est une roche acide composée essentiellement de quartz et de
  mica\footnote{do}.}

  La région est riche en granit (standard), mais il existe des veines de
  minerai de plus forte valeur. Vous consulterez avec attention le rapport de
  votre géologue en chef qui reprend les différents minerais présents sur site.

\subsection{Les nains (standard)}

  \textit{Les nains sont des êtres humanoïdes de petite taille avec une barbe,
  deux bras, deux jambes, un casque et généralement une ou plusieurs choppes de
  bières dans les mains.}

  La direction vous a affecté la meilleure catégorie de personnel pour
  l'exploitation, vous êtes donc à la tête d'une équipe de nains. Le manuel
  relatif à l'exploitation minière, annexé en fin de document, vous donnera de
  plus amples détails. Sachez toutefois que les nains ont certaines\ldots{}
  particularités, comme des \textit{points d'action}, \textit{de déplacement}
  et \textit{de vie}.

  Pour des questions de restrictions budgétaires, le nombre de nains
  (standard) est limité.

\subsection{De la concurrence pour plus de productivité}

  La société a décidé de vous mettre en concurrence avec un autre manager de
  nains, afin d'exploiter au mieux le gisement. Sachez que celui d'entre vous
  qui fera le plus de profit remportera une très belle prime\footnote{dans la
  limite des stocks disponibles} !

  P.R.O.L.O.G.I.N. n'est pas responsable des blessures que subissent ses 
  employés.Elle nie toute implication dans les récentes rumeurs sur les 
  bagarres entre équipes concurrentes, les contusions de pioches relevées 
  n'étant que le produit de malheureux accidents\footnote{Mineur est un 
  métier dangereux, même à plus de 18 ans}.

\newpage

\section{Rapport géologique}

\subsection{Le granit (standard)}

  \textit{Le granit est une roche magmatique plutonique à structure grenue,
  c'est-à-dire entièrement cristallisée, formée par le refroi\ldots{}}

  Bref, comme vous le savez déjà, le granit (standard) est présent en abondace
  sur le site. Cependant, il a une valeur marchande de très exactement 0 pièces
  d'or\footnote{Avec 20\% de taxes, on peut mieux faire en terme de
  rentabilité.}, et requiert un seul coup de pioche pour être cassé.

\subsection{Le minerai}

  En plus du granit (standard), il a été identifié 6 types de minerai
  différents.  Le marché pouvant être particulièrement volatile, leur valeur en
  pièces d'or varie, de même que le nombre de coups de pioche nécessaires pour
  les extraire. Les valeurs dépendent de la mine en question, référez-vous au
  géologue en chef et aux courtiers pour de plus amples informations.

  Voici en attendant une liste de ces minerais, avec un échantillon pour les
  distinguer plus facilement et des approximations de leurs caractéristiques :
  \begin{center}
    \begin{tabular}{|l|Sc|}
      \hline
      Minerai & Échantillon \\
      \hline
      Obsidienne & \cincludegraphics[width=0.5cm]{frames/obsidian.png} \\
      \hline
      Granit & \cincludegraphics[width=0.5cm]{frames/granit.png} \\
      \hline
      Charbon & \cincludegraphics[width=0.5cm]{frames/coal.png} \\
      \hline
      Fer & \cincludegraphics[width=0.5cm]{frames/iron.png} \\
      \hline
      Or & \cincludegraphics[width=0.5cm]{frames/or.png} \\
      \hline
      Diamants & \cincludegraphics[width=0.5cm]{frames/diamonds.png} \\
      \hline
      Émeraudes & \cincludegraphics[width=0.5cm]{frames/emerauld.png} \\
      \hline
      Rubis & \cincludegraphics[width=0.5cm]{frames/ruby.png} \\
      \hline
    \end{tabular}
  \end{center}

\subsection{Divers}

  Le terrain comporte de nombreuses cavités naturelles, formant parfois des
  trous\footnote{des petits trous, encore des petits trous ...} profonds. C'est
  très utiles pour descendre vos nains plus rapidement \ldots{} dans les
  deux sens du termes ! Plus ils chuteront de haut, plus ils se blesseront à
  l'arrivée\footnote{Nous rappelons que P.R.O.L.O.G.I.N. n'est responsable en aucune
  manière des blessures reçues par ses employés, et que toute la charge en
  incombe sur le manager (vous donc).}. Vous trouverez plus de détails
  sur les risques de chute dans le manuel d'exploitation.

  Encore un point de détail : de l'obsidienne a été trouvée en sous-sol. Il
  est inutile de s'en préoccuper, les nains ne peuvent pas la miner\footnote{et
  non pas laminer} avec le matériel qu'ils ont a disposition.

\newpage

\section{Manuel relatif à l'exploitation minière}

  \textit{Version 13, édition 37, opus 42}

\subsection{Reconnaissance du terrain}

  Le standard\footnote{héhéhé} de la compagnie est d'exploiter les gisements
  depuis la surface, en creusant verticalement. L'exploitation à flanc de
  montagne n'est pas pratiquée, sauf exception validée par la direction.

  Le terrain est donc globalement plat, et l'exploitation se fait globalement
  verticalement.

  Conformément aux articles 16 et suivants de la Convention Collective
  Applicable, une taverne est présente en surface de chaque mine. Les mesures
  de sécurité prévoient que les mineurs sans affectation doivent y être
  présents en permanence. Par dérogation au code du travail, la consommation de
  bière pendant les horaires de travail est autorisée, en vertu de ses
  propriétés médicales exceptionnelles : chaque mineur se présentant à la
  taverne récupère instantanément la totalité de ses points de vie\footnote{Ce
  type de réaction n'a été observé dans nos laboratoires que chez les nains
  standard adultes, n'essayez pas chez vous même si vous mesurez moins d'un
  mètre 50.}.

  La taverne abrite également une délégation du département trading, qui
  collecte le minerai extrait.

\subsection{Cordages}

  Le matériel des nains inclut des cordes (standard) illimitées. Pour les
  utiliser il convient de disposer une poulie en tête de cordage, cela
  permettra les actions sur la corde.

  Ces cordes permettent au nains de se déplacer verticalement plus rapidement
  qu'en s'agrippant aux parois et plus surement qu'en chutant.

  Une fois la poulie posée, la corde descend jusqu'au prochain bloc, de plus
  votre équipe de nains s'assurera de prévoir suffisamment de jeu afin que la
  corde puisse toujours se dérouler jusqu'au sol lorsque de nouveaux blocs de
  granite sont creusés en dessous de celle-ci. En revanche, poser une corde est
  une entreprise complexe, qui nécessite la totalité des \textit{points
  d'action} de toute l'équipe combinée.

  De plus les cordes peuvent être actionnées par un autre nain qui n'est pas
  sur la corde. En consommant des points d'actions le nain peut actionner la
  corde dans un sens ou dans un autre ce qui ralentira ou accélérera le 
  mouvement du (ou des) nain sur la corde.

\subsection{Déplacements}

  Le personnel minier peut se déplacer dans deux dimensions: verticalement et
  transversalement, en marchant au sol, en s'agrippant, en
  chutant ou emporté par une corde.

  Un déplacement n'est possible que vers une position libre\footnote{C'est
  évident mais ça va toujours mieux en le disant.}, et nécessite des
  \textit{points de déplacement}. De plus, nous rappelons aux nains que les
  effets de la gravité existe\footnote{Nos scientifiques ont rapporté que les
  nains qui en oubliaient l'existence affichaient un rendement plus faible,
  voir nul.}, nous vous recommandant donc de lire attentivement la section
  dédié.

  Il est également à noter que la promiscuité ne pose pas de problème aux
  mineurs\footnote{C'est un critère de recrutement.}, ils peuvent donc se
  trouver sur une même position. Il est néanmoins nécessaire de se coordonner
  un minimum avant: ainsi seuls les mineurs \textbf{d'une même équipe} se
  tolèrent entre eux.

  Enfin, de par leur activité physique intense, les mineurs ont les épaules
  larges. Ils peuvent sans problème supporter le poids d'un autre mineur. En
  application du paragraphe précédent, un mineur ne gardera jamais un mineur
  qui est dans son équipe sur ses épaules car ils peuvent se coordonner afin de
  se tenir sur la même case.

\subsection{Cas particulier des déplacements verticaux}

  Dans le cas particulier des déplacements verticaux, il est rappelé que la
  gravité existe, et qu'elle attire inexorablement les corps vers le bas. Les
  collisions avec le sol entraînent des dégâts exponentiels avec la hauteur de
  chute, selon la formule ci-dessous.

  \[
    \text{Dégâts} =
    \left\{
      \begin{array}{l l}
        0         & \quad \text{pour $h < 4$}  \\
        2^{h-4} & \quad \text{pour $h >= 4$} \\
      \end{array}
    \right.
  \]

  avec $h$ la hauteur de chute. Si un nain meurt de chute, les dieux vont
  immédiatement s'occuper de faire disparaitre son butin\footnote{Il est
  inscrit dans les registres sacrés:  ``On subit tellement avec les bêtises
  des nains qu'on a bien été forcé d'y mettre une taxe''.}.

  Cependant il est possible d'échapper à ça: un nain agrippé à la paroi, libéré
  de l'emprise de la gravité, peut se déplacer dans toutes les directions, ce
  qui lui demandera plus de \textit{points d'action} que de se déplacement au
  sol. Il y a alors plusieurs manière de se déplacer à distinguer pour un nain
  accroché à la paroi:

  \begin{itemize}
    \item déplacement standard (relativement lent)
    \item déplacement dans une direction occupée par une corde (plus rapide)
    \item rester agrippé sur une case occupée par une corde actionnée par un
      autre nain (ce qui peut faire gagner beaucoup de temps si vous avez
      confiance en l'opérateur de cette corde)
  \end{itemize}

  Vous ne pouvez pas pousser les nains dans les trous (i.e. les putsch sont
  interdits).

\subsection{Traitement du minerai}

  Les minerais précieux doivent être extraits par du personnel qualifié.
  L'extraction consomme des points d'action, et peut nécessiter plusieurs coups
  de pioche en fonction du minerai extrait.

  Comme les nains n'ont pas de problèmes à se tenir côte à côte, il est
  possible de paralléliser l'extraction avec plusieurs mineurs, de manière à
  accélérer celle-ci.

  Une fois extrait, le minerai est stocké temporairement par le personnel. Afin
  d'être pris en compte par le département trading (et être mis en
  sécurité\ldots{}), le minerai doit être rapporté à la taverne.

  Notez que comme les nains ne peuvent porter qu'une quantité limitée de
  minerai\footnote{Tous leurs points de compétences dédiés ayant été investis
  dans le transport de bière.}, ils devront régulièrement faire un retour à la
  taverne pour décharger celui-ci. Si un nain extrait du minerai qu'il ne peut
  pas porter, il détruit immédiatement ce qu'il y a de trop pour qu'il ne tombe
  pas entre de mauvaises mains\footnote{Ami ou ennemi, un nain nai jamais trop
  prudent.}.

\subsection{Cas particulier des rencontres malveillantes}

  Il est rappelé à l'ensemble du personnel que la pioche (standard) qui est
  donnée à chaque nains, peut également être considéré comme une arme de
  quatrième catégorie\ldots{} Son usage à cette fin est explicitement ignoré par
  la direction, à la discrétion de chacun. Dans ce cas, la pioche se manie de
  manière identique à son utilisation standard. À noter que si plusieurs nains
  sont sur la même position et que l'un d'entre eux est attaqué, ils sont tous
  blessés\footnote{Striiike!} !

  Un soin particulier sera accordé aux \textit{points de vie} des intervenants.
  Lorsque ce compteur arrive à zéro, un point de destin\footnote{La direction
  assure qu'un nain aura toujours un point de destin disponible pour lui dans
  cette situation, au risque d'un prélèvement de salaire.} est automatiquement
  utilisé. Le mineur ainsi ressuscité regagne la mine dans la taverne du
  chantier à son prochain tour, mais ayant perdu tout son butin sur le coup
  \ldots{}

  Le nain qui a donné le coup de pioche s'empresse alors de récupérer le butin
  du nain éliminé, dans la limite de la \textit{capacité} qu'il a à le
  transporter, comme expliqué dans la section précédente, il détruira tous ce
  qu'il n'arrive pas à emporter avec lui.
