% SPDX-License-Identifier: GPL-2.0-or-later
% Copyright 2016 Antoine Pietri
% Copyright 2018 Thibault Allançon

\section{Tournois}

\subsection{Tournois intermédiaires}

Afin de vous aider à perfectionner vos algorithmes, des tournois intermédiaires
vous seront proposés. Ces matchs n'ont absolument aucune influence sur le
classement final, mais sont néanmoins à prendre au sérieux, car ils vous
permettront de vous situer par rapport aux autres joueurs, de connaître vos
ennemis, vos points forts et vos faiblesses, et vous donneront des pistes pour
vous améliorer pendant la finale. L'annonce des tournois intermédiaires se fera
environ une heure avant.

À chacun tournois, nous prendrons le dernier champion que chaque candidat aura
envoyé sur le site de soumission pour le faire participer au tournoi, et nous
vous donnerons les résultats ainsi que votre progression dès que les tournois se
seront terminés, avec un récapitulatif de votre progression globale.

Les tournois seront exécutés sur des cartes officielles de notre choix, qui
seront potentiellement amenées à changer au fur et à mesure.

\subsection{Rendu final}

Le tournoi final aura lieu \textbf{samedi à 23~h~42}.

Le rendu final est le seul rendu qui comptera pour le classement. Les mêmes
règles s'appliquent : le dernier champion soumis à l'heure du début du tournoi
sera le champion utilisé pour le tournoi final.

Lors du tournoi final, plusieurs cartes seront ajoutées. Celles-ci resteront
inconnues de tous les joueurs jusqu'à la fin du concours, afin de mesurer
l'adaptabilité de vos algorithmes à des situations inconnues.

Pour le rendu final, nous vous demandons de rajouter des commentaires qui
résument le fonctionnement des différents blocs logiques de votre code, ainsi
qu'un \textbf{commentaire global en haut de votre fichier principal} qui
détaille votre stratégie ainsi que les différents algorithmes que vous avez
employés pour l'implémenter.

\section{Considérations techniques}

Vous disposez d'une seconde (temps réel !) à chaque fois qu'une de vos
fonctions est appelée pour rendre la main. Passé ce délai, votre programme est
tué, le match continue sans vous et vos fonctions ne sont plus appelées. Il
n'est pas possible de revenir en jeu tout simplement parce qu'il n'y a aucun
moyen de rétablir l'état des environnements des langages après une
interruption.  Les limites de mémoire sont faites avec des \texttt{cgroups}, ce
qui fait que l'allocation échouera si vous essayez de dépasser la limite qui
vous est accordée. Cette limite compte aussi la taille de la pile.

D'autres limitations sont appliquées :

\begin{itemize}
    \item le système de fichiers est entièrement en lecture seule ;
    \item seuls \texttt{/usr}, \texttt{/var} et \texttt{/tmp} sont montés ;
    \item vous n'avez pas le droit d'utiliser des processus en parallèle ;
    \item la mémoire est limitée à 500 Mio ;
    \item la taille totale de votre output ne doit pas dépasser 256 Kio (elle
        sera tronquée à partir de cette limite) ;
    \item le temps d'exécution total du processus est limité à 300 secondes de
        temps réel ;
    \item chaque appel de fonction est limité à une seconde de temps réel plus
        500 millisecondes de marge pour prendre en compte le surcoût de
        sérialisation/désérialisation des valeurs depuis et vers les langages
        cibles.
\end{itemize}
